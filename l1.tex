%\documentclass[a4paper, english]{article}

%\usepackage{mystyle}

%\title{Lecture 1: Introduction}
%\author{}
%\date{}


%\begin{document}
%\maketitle{}
The famous \emph{Milnor conjecture} was stated in 1969 by Milnor in his seminal paper \cite{Milnor-quad-forms}. In it he defined the \emph{Milnor $K$-theory} $K_*^M(F)$ of a field $F$ and related this graded ring to Galois cohomology $H_{\mathrm{Gal}}^*$ as well as the associated graded Witt ring 
\[\Gr_I(W(F))\defeq\bigoplus_{n\ge0}I^n(F)/I^{n+1}(F)=\Z/2\oplus I(F)/I^2(F)\oplus\cdots,\]where $I(F)\subseteq W(F)$ is the fundamental ideal. Establishing isomorphisms in degrees 0, 1 and 2, Milnor asked whether this holds in every degree. By the invent of motivic homotopy theory, Voevodsky and his collaborators settled the conjecture in 2003. Every known proof of the Milnor conjecture uses motivic homotopy theory.

\begin{theorem}[Milnor's conjecture]
Suppose $F$ is a field of characteristic different from $2$. Then there exist graded ring isomorphisms
\[\begin{tikzcd}
 & K_*^M(F)/2K^M_*(F)\arrow{ld}[swap]{\cong}\arrow{rd}{\cong} & \\
\Gr_I(W(F))\arrow{rr} & &H^*_{\mathrm{Gal}}(F,\Z/2),
\end{tikzcd}\]
where
\[
K_n^M(F)\defeq (F^\times)^{\otimes n}/\ip{a_1\otimes\cdots\otimes a_n:a_i+a_{i+1}=1\text{ for some }i}
\]is the $n$-th Milnor $K$-group.\end{theorem}

As opposed to the situation in topology, in motivic homotopy theory there are not one but two spheres: The simplicial sphere $S^1$, known to topologists; and the \emph{Tate sphere} $S^\alpha\defeq\G_m$. This results in a bigrading on several important theories in motivic homotopy---most notably on the motivic homotopy groups and motivic cohomology. In fact, in the proof of Milnor's conjecture, the Milnor $K$-theory $K_*^M$ is replaced by motivic cohomology. Furthermore, the Galois cohomology $H_{\mathrm{Gal}}^*$ is replaced by étale motivic cohomology, i.e., motivic cohomology on an étale site.

\section{The idea of motivic homotopy theory}
Techniques from topology are used extensively, but in an algebro-geometric way. Most notably is the use of:
\begin{itemize}
\item Thom-spaces
\item Spectra
\item Steenrod operations, i.e., operations on motivic cohomology
\item Objects representing (co)homology theories
\item The Hopf map, which in the geometric case is induced by the canonical map $\A^2\setminus\{0\}\to\PP^1$. Taking complex points yields the topological Hopf fibration $S^3\to S^2$.
\end{itemize}

In topology, homotopies are parametrized by the unit interval $I=[0,1]$. In motivic homotopy, the algebro-geometric object corresponding to $I$ is the affine line $\A^1\defeq\A^1_F$. Evaluating polynomials in $0$ and $1$ gives us paths. 

\begin{definition}
Let $f,g:X\to Y$ be two morphisms of schemes over $F$. We say that $f$ and $g$ are \emph{elementary $\A^1$-homotopic}, writing $f\simeq g$, if there is a map
\[H:X\times_F\A^1\longrightarrow Y\]
such that
\[H\circ i_0=f,\quad H\circ i_1=g,\]
where $i_j:X\to X\times\{j\}$ $(j=0,1)$ are the canonical inclusions.
\end{definition}
This definition is the ``naive'' lift of the homotopy relation in topology. However, the situation is here quite different: As opposed to topology, the relation of elementary $\A^1$-homotopy is \emph{not} an equivalence relation, because it is not transitive. However, we can consider the equivalence relation generated by elementary $\A^1$-homotopy. We will call two maps in the same equivalence class \emph{$\A^1$-homotopic}.

\subsection{Setting}There are several possible settings for motivic homotopy. One possibility is the following:

Start with the category $\Sm_F$ of smooth schemes of finite type over $F$ (in fact, there is no need to restrict ourselves to working over a field). In order to do homotopy theory we frequently consider quotient spaces. However, the category $\Sm_F$ is poorly behaved under colimits:
\begin{example}
The colimit of the diagram $*\leftarrow\{0,1\}\hookrightarrow\A^1$ is a nodal curve, hence the quotient of two smooth $F$-schemes need not be smooth.
%More severely, by \cite[Appendix B, Example 3.4.2]{Hartshorne}, the quotient of a scheme by a group action need not be a scheme.
\end{example}

We will therefore embed the category $\Sm_F$ in a larger category with better categorical properties. Consider the embedding
\[
\Sm_F\hookrightarrow[\Sm_F^\op,\Set]\eqdef\Pre(\Sm_F).
\]Here $\Pre(\Sm_F)$ is the category of presheaves on $\Sm_F$, i.e., the functor category having as its objects functors $\Sm_F^\op\to\Set$, and the morphisms are natural transformations. The embedding $\Sm_F\hookrightarrow\Pre(\Sm_F)$ is the Yoneda embedding $X\mapsto\Hom_{\Sm_F}(-,X)$.

Since our aim is to find a suitable category for doing homotopy theory, we need to consider a simplicial version of the category $\Pre(\Sm_F)$. Let 
\[\calS\defeq\Delta^\op\Set\defeq[\Delta^\op,\Set]\]
denote the category of simplicial sets. Thus $\Delta$ is the simplex category, whose objects are finite ordered sets
\[[n]\defeq\{0<1<\cdots<n\}\]for $n\ge0$; and the morphisms are maps preserving the ordering. The functor category $\Delta^\op\Set$ is a combinatorial model for topological spaces, in which we can do homotopy theory---i.e., there are model structures on $\Delta^\op\Set$.

Finally, we arrive at the definition of \emph{motivic spaces}: We set
\[
\calMS_F\defeq\Delta^\op\Sm_F=[\Sm_F^\op,\calS].
\]
Note that we have a canonical embedding $\Sm_F\hookrightarrow\calMS_F$ by considering a scheme $X$ as a constant motivic space. We think of a motivic space as the analog in motivic homotopy theory to a topological space in ordinary homotopy.

\subsection{Algebro-geometric versions of topological simplices}

\begin{example}[Simplicial sets]
A topological space $X$ yields a singular chain complex. 

The standard topological $n$-simplex is given by
\[
\Delta_n\defeq\left\{(x_0,\dots,x_n)\in\R^{n+1}:\sum_i x_i=1,x_i\ge0\right\}.
\]The face maps $d^i:\Delta_{n-1}\to\Delta_n$ inserts a 0 in the $i$th coordinate, and the degeneracy maps $s^i:\Delta_{n+1}\to\Delta_n$ adds the coordinates $x_i$ and $x_{i+1}$. Geometrically, $d^i$ inserts $\Delta_{n-1}$ as the $i$th face of $\Delta_n$, and $s^i$ projects $\Delta_{n+1}$ onto the $n$-simplex orthogonal to its $i$th face.


Define the simplicial set $\Sing(X)$ as
\[\Sing(X)_n\defeq\Hom_\Top(\Delta_n,X);\]
the elements of $\Sing(X)_n$ are called $n$-simplices of $X$. Precomposing with $d^i$ or $s^i$ yields respectively face maps 
\[d_i:\Sing(X)_{n+1}\to\Sing(X)_n\]
and degeneracy maps
\[
s_i:\Sing(X)_{n-1}\to\Sing(X)_n.\]

Using the functor $\Sing$ we can factor the $n$-th singular homology functor $H_n(-;\Z)$ as
\[\begin{tikzcd}
H_n(-;\Z):\Top\arrow{r}{\Sing} & \calS\arrow{r}{\Z} & \Delta^\op\Ab\arrow{r}{\sum(-1)^id_i} & \Ch_\Z\arrow{r}{H_n} & \Ab.
\end{tikzcd}\]
Here $\Z:\calS\to\Delta^\op\Ab$ is the free functor in each degree, and $\Ch_\Z$ is the category of chain complexes.
\end{example}

We now have the diagram
\[\begin{tikzcd}\Delta\arrow{r}{\text{Yoneda}}\arrow{dr}[swap]{\Delta_\bullet} & \calS\arrow{d}{\re} \\
& \Top\end{tikzcd}\]
where $\Delta_\bullet([n])\defeq\Delta_n$, and the ``realization functor'' $\re$ is the left Kan extension of $\Delta_\bullet$ along the Yoneda embedding. Defining $\re(\Delta(-,[n]))\defeq\Delta_n$ determines the functor. Furthermore, the functor $\re$ has $\Sing$ as its right adjoint,
\[
\re:\calS\rightleftarrows\Top:\Sing.
\]
We can put a model structure on $\calS$ such that this adjoint pair becomes a Quillen equivalence.

\begin{example}[Standard simplicial ring]
For any $n\ge0$, let
\[
F_n\defeq \dfrac{F[x_0,\dots,x_n]}{\p*{\sum_ix_i-1}}.
\]
For each $i$, define face maps $F_{n+1}\to F_n$ by $x_i\mapsto0$, and degeneracy maps $F_{n-1}\to F_n$ by $x_i\mapsto x_i+x_{i+1}$. Define
\[
\Delta_F^n\defeq\Spec F_n.
\]There is a noncanonical isomorphism $\A^n_F\cong\Delta_F^n$. Letting $n$ vary we obtain a functor
\[
\Delta_F^\bullet:\Delta\to\Sm_F,
\]i.e., $\Delta_F^\bullet$ is a cosimplicial object.
\end{example}

\begin{example}
As a first attempt to define singular homology for schemes, we can try to follow the recipe from topology. Suppose $F=\Q$, so that $\Delta_\Q^0=\Spec\Q$. Consider the affine scheme $\Spec\Q(\sqrt{-1})$. Since there are no ring maps $\Q(\sqrt{-1})\to\Q$,
\[
\Hom_{\Sm_\Q}(\Delta_\Q^0,\Spec\Q(\sqrt{-1}))=\varnothing.
\]This approach would therefore yield the uninteresting result $H_0(\Q(\sqrt{-1});\Z)=0$.
\end{example}
The problem of defining singular homology for schemes was solved by Suslin-Voevodsky. They proved an algebraic version of the Dold-Thom theorem, which states that if $X$ is a pointed CW-complex, there is a weak equivalence
\[\begin{tikzcd}
\Sym^\infty(X)\arrow{r}{\sim} & \re(\Z\{\Sing(X)_\bullet\}),
\end{tikzcd}\]
where $\Sym^\infty=\colim_n\Sym^n(X)$.

Suslin-Voevodsky replaces the free functor $\Z:\calS\to\Delta^\op\Ab$ with the functor $\Cor_F$ of correspondences, which we will come back to shortly. For $X$ a smooth $F$-scheme, define
\[
H^{\text{Suslin}}_p(X;\Z)\defeq H_p(\Cor(\Delta_F^\bullet,X))
\]as the \emph{Suslin homology} of $X$ over $F$. This homology theory has several good properties. For example, if $X$ is a complex variety,
\[
H^{\text{Suslin}}_p(X;\Z/n)\cong H_p^{\text{sing}}(X^{\text{an}};\Z/n),
\]where $X^{\text{an}}$ is the analytic space associated with $X$.

\subsection{Correspondences}
Correspondences are in some sense ``more sensitive'' than taking free abelian groups via the functor $\Z:\calS\to\Delta^\op\Ab$ mentioned above.
\begin{definition}
Let $X,Y\in\Sm_F$. An \emph{elementary correspondence} is a closed irreducible subset $W\subseteq X\times Y$, with a finite surjective map $W\to X$. Define $\Cor_F(X,Y)$ as the free abelian group on elementary correspondences.
\end{definition}
We then have $\Cor(X,\Spec F)=\bigoplus_{X_i}\Z$, where the sum ranges over all connected components $X_i$ of $X$.

Correspondences give rise to a category of motives over $F$,
\[\begin{tikzcd}\Sm_F\arrow{rr}\arrow{rd} & & \calMS_F\arrow{ld}\\
& \DM_F &
\end{tikzcd}\]
Here $\DM_F$ is Voevodsky's derived category of motives, whose objects are chain complexes of presheaves on $\Sm_F$ with transfers.

\section{Motives}
Let $\calV_F$ denote the category of smooth projective varieties over $F$. Grothendieck and his collaborators speculated on the existence of a category $\calM_F$ of so-called pure motives, through which any Weil cohomology theory should factor through.
\begin{example}
The following are examples of Weil cohomology theories:
\begin{itemize}
\item Betti cohomology: For $F\subseteq\C$, $H^*_B(X)\defeq H^*_{\text{sing}}(X^{\text{an}};\Q)$.
\item $\ell$-adic cohomology: For $\ell$ different from the characteristic of $F$, \[H_\ell^*(X)\defeq H^*_{\text{ét}}(X;\Q_\ell)=\varprojlim H_{\text{ét}}^*(X;\Z/\ell^n)\otimes_{\Z_\ell}\Q_\ell.\]
\item Crystalline cohomology $H^*_{\text{crys}}(X;\Q_{\text{char}F})$.
\item Algebraic de Rham cohomology: For $\text{char}(F)=0$, $H^*_{\text{dR}}(X)\defeq\mathbf H^*(X;\Omega_{X/F}^\bullet)$, where $\Omega^\bullet_{X/F}$ is the de Rham complex of $X$, and $\mathbf H$ denotes hypercohomology.
\end{itemize}
We see that $H^*_?$ takes values in different vector spaces, and that some theories impose restrictions on $F$. However, there are perhaps more similarities than differences between the above theories:
\begin{enumerate}
\item All cohomology theories are contravariant functors.
\item $\dim H^0_?(X)=1=\dim H^{2\dim X}_?(X)$.
\item $H^i_?(X)=0$ for $i<0$ or $i>2\dim X$.
\item $H^*_?$ satisfies Poincaré duality and Künneth theorems.
\end{enumerate}
A \emph{pure motive} should thus mean a universal theory satisfying the properties 1-4 above.
\end{example}

\begin{theorem}[Grothendieck]
The category $\calM_F$ exists if and only if the standard conjecture on algebraic cycles holds.
\end{theorem}
The main idea for constructing a category of motives is as follows. We start out by a geometric category of some kind, e.g., $\calV_F$ or $\Sm_F$, and then form a linear category (i.e., the Hom-objects are abelian groups). This linear category can be for example an abelian tensor-category; a triangulated category or a stable $\infty$-category. For example, Voevodsky's $\DM_F$ is a derived version of $\calM_F$. 

This is inspired by the Beilinson conjectures, which says that there exist complexes of Zariski sheaves $\Z(n)$, for $n\ge0$, such that:
\begin{enumerate}
\item $\Z(0)=\Z$; $\Z(1)=\calO^*[-1]$.
\item $\mathbf H^n(F;\Z(n))=H^m_M(F;\Z(n))\cong K_n^M(F)$.
\item $H^{2n}(X;\Z(n))\cong\text{CH}^n(X)$, where $\text{CH}^n(X)$ is the Chow group of $X$.
\item $H^p(X;\Z(n))=0$ for all $p<0$ (Beilinson-Soulé vanishing conjecture).
\item There is a spectral sequence
\[
E_2^{p,q}=H^{p,q}(X;\Z(-q))\implies K_{-p-q}(X),
\]where the righthand side is algebraic $K$-theory of $X$.
\item $\Z(n)\otimes^{\mathbf L} \Z/\ell\cong \tau_{\le n}R\pi_*\mu_\ell^{\otimes n}$, where $\pi:(\Sm_F)_{\text{ét}}\to(\Sm_F)_{\text{Zar}}$ is the forgetful functor, $\tau$ is a truncation functor and $\mu_\ell^{\otimes n}$ is the étale sheaf of roots of unity on $\Sm_F$. This is known as the Beilinson-Lichtenbaum conjecture.
\end{enumerate}
The Beilinson-Lichtenbaum conjecture is equivalent to the Bloch-Kato conjecture, which contains Milnor's conjecture as the special case $\ell=2$. Voevodsky et al. have proven every point above except 4.


















%\bibliography{ref}
%\bibliographystyle{alpha}

%\end{document}